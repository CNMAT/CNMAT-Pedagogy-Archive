\documentclass[11pt, letterpaper]{article}
\usepackage{fullpage}
\usepackage{graphicx}
\usepackage{paralist}
\usepackage{verbatim}
\usepackage{ifpdf}
\usepackage{MnSymbol}
\usepackage{hyperref}

\newcommand{\ctitle}[1]{\title{CNMAT Max/MSP Summer Workshop \the\year\\Lab Assignment #1}}

\def\imagedir {images/}

% counter to protect against nested \HCode calls
\newcounter{hcode-counter}

% tutorial link
% \newcommand{\tut}[3]{\href{http://www.cycling74.com/docs/max5/tutorials/#1/#2.html}{#3}}
\newcommand{\tut}[3]{\href{http://cycling74.com/docs/max6/dynamic/c74_docs.html\##2}{#3}}
\newcommand{\maxtut}[2]{\tut{max-tut}{#1}{#2}}
\newcommand{\msptut}[2]{\tut{msp-tut}{#1}{#2}}

% max object
\newcommand{\m}[2]{[#1~#2]}
%\newcommand{\mdoc}[2]{[\href{http://www.cycling74.com/docs/max5/refpages/max-ref/#1.html}{#1}~#2]}
\newcommand{\mdoc}[2]{[\href{http://cycling74.com/docs/max6/dynamic/c74_docs.html\##1}{#1}~#2]}

% msp object
\newcommand{\msp}[2]{
\ifpdf
\m{#1$\sim$}{#2}
\else
\ifnum\value{hcode-counter} > 0
\m{#1~ #2}
\else
\HCode{[#1&\#126; #2]}
\fi
\fi
}

\newcommand{\mspdoc}[2]{
\ifpdf
%[\href{http://www.cycling74.com/docs/max5/refpages/msp-ref/#1~.html}{#1}$\sim$~#2]
[\href{http://cycling74.com/docs/max6/dynamic/c74_docs.html\##1~}{#1}$\sim$~#2]
\else
%[\href{http://www.cycling74.com/docs/max5/refpages/msp-ref/#1~.html}{#1}\HCode{&\#126;}~#2]
[\href{http://cycling74.com/docs/max6/dynamic/c74_docs.html\##1~}{#1}\HCode{&\#126;}~#2]
\fi
}

% message box
\newcommand{\msg}[1]{
\ifpdf
$\ullcorner\textrm{#1}\ulrcorner$
\else
\ifnum\value{hcode-counter} > 0
<table style="display:inline-table; vertical-align:-25\%" cellpadding="0" cellspacing="0" border="0">
<tr>
<td style="font-size:75\%;">\&\#x2308;</td>
<td rowspan="2">#1</td>
<td style="font-size:75\%;">\&\#x2309;</td>
</tr>
<tr>
<td style="font-size:75\%;">\&\#x230a;</td>
<td style="font-size:75\%;">\&\#x230b;</td>
</tr>
</table>
\else
\stepcounter{hcode-counter}
\HCode{
<table style="display:inline-table; vertical-align:-25\%" cellpadding="0" cellspacing="0" border="0">
<tr>
<td style="font-size:75\%;">&\#x2308;</td>
<td rowspan="2">#1</td>
<td style="font-size:75\%;">&\#x2309;</td>
</tr>
<tr>
<td style="font-size:75\%;">&\#x230a;</td>
<td style="font-size:75\%;">&\#x230b;</td>
</tr>
</table>
}
\setcounter{hcode-counter}{0}
\fi
\fi
}

% variable
\newcommand{\var}[1]{
\ifpdf
$<$#1$>$
\else
\ifnum\value{hcode-counter} > 0
&lt#1&gt
\else
\stepcounter{hcode-counter}
\HCode{&lt#1&gt}
\setcounter{hcode-counter}{0}
\fi
\fi
}

% insert a screen shot
\newcommand{\sshot}[3]{
\ifpdf
\begin{center}\includegraphics[width=#1]{#3}\end{center}
\else
\HCode{<div style="text-align: center;"><img src="\imagedir#3.png" width=#2></div>}
\fi
}

\newcommand{\gt}{Gametrak}

\graphicspath{{\imagedir}}

% \title{CNMAT Max/MSP Summer Class 2011\\Lab Assignment 1}
\ctitle{1}
\date{\today}

\begin{document}
\maketitle

\section*{Summary}
In this lesson, we will use data from the \gt{} to control the pitch and volume
of a virtual \href{http://en.wikipedia.org/wiki/Theremin}{Theremin}.  We will begin
by visualizing the data coming from the \gt{} to figure out what it is and 
what we will need to do to it to make it suitable to control our instrument.

\subsection*{Topics}
Data visualization (printing, number boxes, multislider), scaling.

\subsection*{Objects Introduced}
% objects:  button, toggle, flonum, number, message, comment, o.message, gate, +, -, *, /, *~, scale, gain~, multislider
\mdoc{button}{}, \mdoc{toggle}{}, \mdoc{flonum}{}, \mdoc{number}{}, \mdoc{message}{}, \mdoc{comment}{},
\mdoc{print}{}, \mdoc{o.message}{}, \mdoc{gate}{}, \mdoc{plus}{} (+), \mdoc{minus}{} (-), \mspdoc{times}{} \mdoc{times}{} (*), \mdoc{div}{} (/),
\mdoc{scale}{}, \mspdoc{gain}{}, \mdoc{multislider}{}

\subsection*{Relevant Tutorials}
\subsubsection*{Basic}
\begin{enumerate}
\item \maxtut{basicchapter01}{Hello}
\item \maxtut{basicchapter03}{Numbers and Lists}
\item \maxtut{basicchapter06}{Simple Math}
\item \maxtut{basicchapter21}{Controlling Data Flow}
\end{enumerate}
\subsubsection*{Data}
\begin{enumerate}
\item \maxtut{datachapter02}{Data Scaling}
\end{enumerate}
\subsubsection*{MSP Topic}
\begin{enumerate}
\item \msptut{mspdigitalaudio}{How Digital Audio Works}
\end{enumerate}


\section{Visualization}

\begin{enumerate}
\item Start by turning on the gametrack input at the top of the patch.  
Notice that nothing happens ;)  This is often the case when working with Max---we 
have some operation that is happening, but we can't see what it is or what it is doing.

We can assume that the object called \m{o.io.gametrak}{} is sending out data that is being 
processed by the four \m{o.route}{} objects below it.  The first thing we should do is 
verify that \m{o.io.gametrak}{} is indeed sending out data, and see what kind of data it is.  

\item Hook up the right inlet of an \mdoc{o.message}{} object to the outlet of the \m{gate}{} object and look at the data.
(If the Max window is not visible on your monitor, bring it forward by 
choosing ``Max Window'' under the ``Window'' menu.)

\sshot{4in}{800}{print_gametrak}

Take a close look at the messages in the \mdoc{o.message}{} box (turn off the Gametrak input if the 
messages are being printed too fast).  Notice that the messages coming from \m{o.io.gametrak}{} 
are in the form /gametrak/left/x \var{number}, or /gametrak/right/z \var{number}.  
What do you think the \m{o.route}{} objects are doing in the patch?


\item Display the values coming in from the \gt{} by connecting the three leftmost outlets of the
two \m{o.route}{/x /y /z} objects to number boxes.
You have a choice between two number boxes:  one that displays floating point (decimal) 
numbers, and one that displays integers.  Based on what you've seen printed to the \mdoc{o.message}{} box, 
which one do you think is appropriate here and what happens if you use the wrong one?

\sshot{4in}{600}{number_boxes}

As you move the two joysticks, you should see the numbers change rapidly.  
While useful for verifying that data is actually flowing through the patch, 
the numbers are changing so fast that it can be difficult to track them.


\item Try visualizing the numbers in the number boxes using the \mdoc{multislider}{} object.  
(HINT:  \mdoc{multislider}{} has a couple of different modes that are useful for displaying data.  
In particular, it has a mode called ``pointscroll''--check out the help file.)

\sshot{4in}{600}{multisliders}

%% A single \m{multislider}{} object is capable of visualizing multiple parameters at once.  
%% Try packing the /x, /y, and /z data from the left \m{o.route /x /y /z}{} object into a 
%% list with the \m{pack} object and sending that into \m{multislider}{} (don't forget to 
%% check out the help file for the \m{pack}{} object\ldots).

\end{enumerate}





\section{Sonification}

\begin{enumerate}
\item Start by turning on the audio system by clicking on the speaker icon 
(\mspdoc{ezdac}{}) at the bottom of the patch.  \textbf{Slowly} turn up the volume by 
dragging the slider \mspdoc{gain}{} above the speaker.  You should hear a sine wave 
tuned to 440. Hz, or A above middle C.

We want to control the frequency ({\em pitch}) and amplitude ({\em volume}) of that 
sine wave using the joysticks of the Gametrak.  

\item Pull the left joystick straight up away from the Gametrak---pull it as far 
as you can without standing and without winding the cord around anything.  
Now take note of the /z parameter of the left \m{o.route}{} object---this will 
be the maximum value of our range (probably somewhere around 0.3).

\item Now let's scale the range of the /z parameter you just recorded to 
a reasonable range of audible frequencies like 200--3500 Hz.  
We will do this by using the \mdoc{scale}{} 
object which takes four arguments:
  \begin{inparaenum}[(1)]
    \item minimum input value
    \item maximum input value
    \item minimum output value
    \item maximum output value.
  \end{inparaenum}
So, if we want to scale input values that have a range from 0--.3 to 200--3500, 
we should create a scale object with the arguments \mdoc{scale}{0.0 0.3 200. 3500.} 
(remember to use floating-point arguments).

Verify that the scaling is working properly by using a number box attached to the 
output of the \mdoc{scale}{} object.

\sshot{4in}{600}{z_scale}

\item Once you have verified that the scaling is working properly, 
connect the outlet of the \mdoc{scale}{} object to the inlet of the \msp{cycle~}{} object.



\item Now let's control the volume with the Gametrak.  Connect the /z parameter of the 
right joystick to the right inlet of the \msp{*}{} (\mspdoc{times}{}) object after the \msp{cycle~}{}
object and before the \mspdoc{gain}{} object.  

\sshot{7in}{500}{volume}


\end{enumerate}

\end{document}
