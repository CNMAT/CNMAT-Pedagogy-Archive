\documentclass[11pt, letterpaper]{article}
\usepackage{fullpage}
\usepackage{graphicx}
\usepackage{paralist}
\usepackage{verbatim}
\usepackage{ifpdf}
\usepackage{MnSymbol}
\usepackage{hyperref}

\newcommand{\ctitle}[1]{\title{CNMAT Max/MSP Summer Workshop \the\year\\Lab Assignment #1}}

\def\imagedir {images/}

% counter to protect against nested \HCode calls
\newcounter{hcode-counter}

% tutorial link
% \newcommand{\tut}[3]{\href{http://www.cycling74.com/docs/max5/tutorials/#1/#2.html}{#3}}
\newcommand{\tut}[3]{\href{http://cycling74.com/docs/max6/dynamic/c74_docs.html\##2}{#3}}
\newcommand{\maxtut}[2]{\tut{max-tut}{#1}{#2}}
\newcommand{\msptut}[2]{\tut{msp-tut}{#1}{#2}}

% max object
\newcommand{\m}[2]{[#1~#2]}
%\newcommand{\mdoc}[2]{[\href{http://www.cycling74.com/docs/max5/refpages/max-ref/#1.html}{#1}~#2]}
\newcommand{\mdoc}[2]{[\href{http://cycling74.com/docs/max6/dynamic/c74_docs.html\##1}{#1}~#2]}

% msp object
\newcommand{\msp}[2]{
\ifpdf
\m{#1$\sim$}{#2}
\else
\ifnum\value{hcode-counter} > 0
\m{#1~ #2}
\else
\HCode{[#1&\#126; #2]}
\fi
\fi
}

\newcommand{\mspdoc}[2]{
\ifpdf
%[\href{http://www.cycling74.com/docs/max5/refpages/msp-ref/#1~.html}{#1}$\sim$~#2]
[\href{http://cycling74.com/docs/max6/dynamic/c74_docs.html\##1~}{#1}$\sim$~#2]
\else
%[\href{http://www.cycling74.com/docs/max5/refpages/msp-ref/#1~.html}{#1}\HCode{&\#126;}~#2]
[\href{http://cycling74.com/docs/max6/dynamic/c74_docs.html\##1~}{#1}\HCode{&\#126;}~#2]
\fi
}

% message box
\newcommand{\msg}[1]{
\ifpdf
$\ullcorner\textrm{#1}\ulrcorner$
\else
\ifnum\value{hcode-counter} > 0
<table style="display:inline-table; vertical-align:-25\%" cellpadding="0" cellspacing="0" border="0">
<tr>
<td style="font-size:75\%;">\&\#x2308;</td>
<td rowspan="2">#1</td>
<td style="font-size:75\%;">\&\#x2309;</td>
</tr>
<tr>
<td style="font-size:75\%;">\&\#x230a;</td>
<td style="font-size:75\%;">\&\#x230b;</td>
</tr>
</table>
\else
\stepcounter{hcode-counter}
\HCode{
<table style="display:inline-table; vertical-align:-25\%" cellpadding="0" cellspacing="0" border="0">
<tr>
<td style="font-size:75\%;">&\#x2308;</td>
<td rowspan="2">#1</td>
<td style="font-size:75\%;">&\#x2309;</td>
</tr>
<tr>
<td style="font-size:75\%;">&\#x230a;</td>
<td style="font-size:75\%;">&\#x230b;</td>
</tr>
</table>
}
\setcounter{hcode-counter}{0}
\fi
\fi
}

% variable
\newcommand{\var}[1]{
\ifpdf
$<$#1$>$
\else
\ifnum\value{hcode-counter} > 0
&lt#1&gt
\else
\stepcounter{hcode-counter}
\HCode{&lt#1&gt}
\setcounter{hcode-counter}{0}
\fi
\fi
}

% insert a screen shot
\newcommand{\sshot}[3]{
\ifpdf
\begin{center}\includegraphics[width=#1]{#3}\end{center}
\else
\HCode{<div style="text-align: center;"><img src="\imagedir#3.png" width=#2></div>}
\fi
}

\newcommand{\gt}{Gametrak}


\graphicspath{{\imagedir}}

\ctitle{2}
% \title{CNMAT Max/MSP Summer Class 2011\\Lab Assignment 2}
\date{\today}

\begin{document}
\maketitle

\section*{Summary}
In this lesson, you will expand on the Theremin you built in lesson 1 by adding
vibrato as a parameter that you can control with the \gt, adding a second
voice, and adding distortion that increases with the volume.

\subsection*{Topics}
Polyphony, encapsulation, abstraction, reuse, parameters.

\subsection*{Objects Introduced}
% patcher, osc-route, tanh~, *~, +~
\mdoc{patcher}{}, \mspdoc{tanh}{}, \mspdoc{times}{}, \mspdoc{plus}{}

\subsection*{Relevant Tutorials}
\subsubsection*{Basic}
\begin{enumerate}
\item \maxtut{basicchapter14}{Encapsulation}
\item \maxtut{basicchapter15}{Abstraction}
\item \maxtut{basicchapter20}{Presentation Mode}
\item \maxtut{basicchapter21}{Controlling Data Flow}
\end{enumerate}
\subsubsection*{MSP}
\begin{enumerate}
\item \msptut{mspchapter01}{Test Tone}
\item \msptut{mspchapter02}{Adjustable Oscillator}
\item \msptut{mspchapter03}{Wavetable Oscillator}
\item \msptut{mspchapter10}{Vibrato and FM}
\end{enumerate}

\section{Vibrato}

Vibrato is an expressive musical device that is created by slowly oscillating the pitch 
being played (\href{http://en.wikipedia.org/wiki/Theremin}{Theremin} virtuoso 
\href{http://en.wikipedia.org/wiki/Clara_Rockmore}{Clara Rockmore} (1911-98) was famous for her beautiful vibrato).  
Of course, you can add vibrato by simply shaking the part of the \gt{} that controls 
the frequency of our Theremin, but since it takes many years of practice to be able to control smooth, 
even vibrato, we'll extend our patch to do it for us.

The \msp{vib}{} object takes two parameters (see the help file for a demonstration):  
\begin{inparaenum}[(1)]
  \item the frequency in Hz around which the output will oscillate (left inlet), and
  \item a parameter between 0--1 that controls both the speed and depth of 
the oscillation (right inlet).
\end{inparaenum}

We'll wire the \msp{vib}{} object in between the \mdoc{scale}{} and \mspdoc{cycle}{} objects and control the 
speed and depth of the oscillation using the \gt.

\begin{enumerate}
\item Disconnect the \mspdoc{cycle}{} object from the \mspdoc{gain}{} object.
\item Connect the outlet of the \mdoc{scale}{} object to the left inlet of the \msp{vib}{} object.
\item Connect the outlet of the \msp{vib}{} object to the left inlet of the \mspdoc{cycle}{} object.
\item Connect the outlet of the \msp{cycle~}{} object to the left inlet of the \msp{*}{} object.
\item Now connect the /y value of the left \m{o.route}{/x /y /z} object to the right inlet of the \msp{vib}{} object.
Even though the controller is sending out values between -1--1, we don't have to change anything because
the \msp{vib}{} object takes the absolute value of the parameters that enter through the right inlet.

\sshot{2.5in}{400}{vib}

\end{enumerate}

%% \section{Smoothing}

%% The data that comes from the \gt{} can be a little noisy and can add audible jitter to our 
%% Theremin.  This zipper-like effect is particularly noticeable if you change frequencies rapidly.  
%% There are many ways to smooth data in Max, two of which are implemented in the \m{smoother}{}
%% and \msp{smoother} objects.  \m{smoother} is constructed to operate on the data going into the \msp{cycle~}{}{} object, 
%% while \msp{smoother} is designed to operate on the output of the \msp{cycle~}{} object.  
%% You need only add one of them to your patch, but you may want to try each of them to see if one works better than the other.

%% 1.  Either wire the \m{smoother} object between the scaled frequency coming from the \gt{} and the \msp{vib}{}{} object,

%% --OR--

%% wire the \msp{smoother} object between the \msp{vib} object and the \msp{cycle~}{} object.

%% For those feeling adventurous, try to write your own smoother (HINT:  \m{smoother} uses \m{zl stream}{}, 
%% \m{zl sum}, and \m{/}, while \msp{smoother} uses \msp{slide}{}{}.  If we haven't covered those 
%% objects in class, have a look at their help files to see what they do.).

%% \sshot[2.5in]{smooth}

\section{Polyphony}

Now our Theremin has three parameters that we are controlling with the \gt, but we're not making very efficient
use of our controller.  In this section, we'll remap the three parameters of the Theremin to the left 
joystick and make a copy of the instrument that we can control with the right joystick.

\begin{enumerate}
\item Rewire your patch so that the /x, /y, and /z parameters of the left joystick control the frequency, 
gain, and vibrato as follows:

\begin{center}
  \begin{tabular}{|l|l|}
    \hline
    \gt{} parameter & Theremin parameter\\
    \hline
    \hline
    /x & frequency\\
    \hline
    /y & vibrato\\
    \hline
    /z & gain\\
    \hline
  \end{tabular}
\end{center}

Since we had been controlling the frequency with the /z parameter which has a range
of 0--1 and we will now be controlling the frequency with the /x parameter which 
has a range of -1--1, we will need to change the arguments of the \mdoc{scale}{} object
so that the first two are -1--1.

\item Now make a copy of all of the objects between the bottom \m{o.route}{/x /y /z} object 
and the \mspdoc{gain}{} object.

\sshot{4in}{400}{copy}

\item Wire up the copy so that the right joystick controls the three parameters of the 
second voice the same way that the left one does. 

\end{enumerate}

\section{Brightness}\label{brightness}

We can improve our Theremin increasing the brightness of the tone as the volume increases.  
Increased brightness occurs when harmonics are added to our sine wave, which we can do by gradually 
transforming our nice round sine wave into a square wave.  One way to do this is to increase its
amplitude and send its output through a \mspdoc{tanh}{} object.  
See the file called brightness.maxpat in the demos folder for a demonstration of how this works.  

\begin{enumerate}

\item Start wiring a \mspdoc{tanh}{} object between the \msp{*}{} (\mspdoc{times}{}) object and the 
\mspdoc{gain}{} object for the left voice.

\item Now we want the /z parameter from the \gt{} that is controlling the volume to have
a greater range so that the signal saturates.  Multiply the /z parameter by 4.

\sshot{4in}{400}{tanh}

\end{enumerate}

\section{Abstraction}\label{abstraction}

We would now like to apply the change that we've made to the left joystick 
(part \ref{brightness} of today's lab) 
to the right joystick.  There are at least two ways to implement this change 
for the second voice that we want to avoid: 
\begin{inparaenum}[(1)]
\item we can simply repeat the necessary steps, or we can 
\item delete the right side and copy the objects from the left and rewire them.
\end{inparaenum}

While neither of these scenarios may seem like a problem, try to imagine what 
it would be like if your patch contained 20 voices each with 20 objects.  
As we continue to copy and paste groups of objects that represent some sort
of functionality, we quickly run into the need to {\em abstract} that 
functionality in a way that would allow us to make the change to one
part of the patch and have that change propagate through all {\em instances}
of that abstraction.

\begin{enumerate}
\item Select the objects in the left part of the patch that represent
the Theremin voice (starting with \mdoc{scale}{} and including \mspdoc{tanh}{}).

\item Under the ``Edit'' menu, choose ``Encapsulate.''

\sshot{4in}{600}{encapsulate_menu}

The objects that you had selected have now been encapsulated in an object called \m{p}{}, 
which is short for \mdoc{patcher}{}.  We call this object a ``subpatch'' or a ``subpatcher.''

\sshot{2.5in}{400}{subpatch}

\item Double-click on the \m{p}{} object that was just created.  You will see a window that contains 
all of the objects that you had selected still wired up in the same way.  

\item Now save a copy of that new patch (by choosing ``Save As...'' from under the ``Edit'' menu).
Save it in the directory for today's lab---the same directory where the main Max patch
you are editing is.  You can call it anything you like (maybe Theremin.maxpat would be a good choice...).

\item Now close the subpatch window and, back in the main patch, 
replace the [p] object with the name of the patch you just saved,
omitting  the .maxpat extension.  So, if you saved your patch with the title Theremin.maxpat, 
you should make an object called \m{Theremin}{}.  

\sshot{2.5in}{400}{abstraction}

There is no need to rewire your patch---since the subpatch and the abstraction you 
just made have the same number of inlets and outlets, Max will simply replace
the subpatch with your \m{Theremin}{} abstraction, keeping the connections
in tact.

\item Now, make a copy of the \m{Theremin}{} abstraction in the patch and replace the 
objects that constitute the second voice with the \m{Theremin}{}.

\sshot{2.5in}{400}{polyphony}

\end{enumerate}

\section{Parameters}

The abstraction you made in part \ref{abstraction} allows you to make a 
change to the file you saved on disk
(Theremin.maxpat) and to have that change take effect wherever it is used.
Try it:

\begin{enumerate}
\item Open the file Theremin.maxpat on your hard drive and make a change to the 
parameters of the \mdoc{scale}{} object and save the patch.

\item Now, go back to the main patch and double click on both instances of 
the \m{Theremin}{} abstraction---you should see the changes you made.

But what if we want to change the parameters of the \mdoc{scale}{} object
in the left instance of the \m{Theremin}{} to \mdoc{scale}{-1. 1. 200. 1000.} 
and those of the right instance to \m{scale}{-1. 1. 750. 3500.} so that we have a 
low and a high instrument?  For this, we need to expose these parameters so they can 
be accessed from the patch that contains the instances.  

\item Open the Theremin.maxpat patch on the hard disk and add two inlets---you can 
do this by creating a new object and typing ``inlet'' into it, or by selecting, copying
and pasting two of the inlets that already exist in the patch.

\item Now connect the left inlet you made to the fourth inlet of the \mdoc{scale}{} object
so that numbers coming in through that inlet control the output minimum value, and 
connect the right inlet you made to the fifth inlet of the \mdoc{scale}{} object.

\sshot{2.5in}{400}{five_outlets}

\item Save and close Theremin.maxpat.  Notice that the two instances of the 
abstraction in the main patch have been updated with two new inlets.

\item Now we can control the minimum and maximum frequencies that the 
\gt's joysticks are mapped to.  Connect number boxes to the new inlets of \m{Theremin}{}
and set the frequencies to 200 and 1000 for the left and 750 and 3500 for the right.

\sshot{4in}{600}{params}

\end{enumerate}

\end{document}
