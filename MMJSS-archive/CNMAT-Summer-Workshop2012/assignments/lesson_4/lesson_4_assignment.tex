\documentclass[11pt, letterpaper]{article}
\usepackage{fullpage}
\usepackage{graphicx}
\usepackage{paralist}
\usepackage{verbatim}
\usepackage{ifpdf}
\usepackage{MnSymbol}
\usepackage{hyperref}

\newcommand{\ctitle}[1]{\title{CNMAT Max/MSP Summer Workshop \the\year\\Lab Assignment #1}}

\def\imagedir {images/}

% counter to protect against nested \HCode calls
\newcounter{hcode-counter}

% tutorial link
% \newcommand{\tut}[3]{\href{http://www.cycling74.com/docs/max5/tutorials/#1/#2.html}{#3}}
\newcommand{\tut}[3]{\href{http://cycling74.com/docs/max6/dynamic/c74_docs.html\##2}{#3}}
\newcommand{\maxtut}[2]{\tut{max-tut}{#1}{#2}}
\newcommand{\msptut}[2]{\tut{msp-tut}{#1}{#2}}

% max object
\newcommand{\m}[2]{[#1~#2]}
%\newcommand{\mdoc}[2]{[\href{http://www.cycling74.com/docs/max5/refpages/max-ref/#1.html}{#1}~#2]}
\newcommand{\mdoc}[2]{[\href{http://cycling74.com/docs/max6/dynamic/c74_docs.html\##1}{#1}~#2]}

% msp object
\newcommand{\msp}[2]{
\ifpdf
\m{#1$\sim$}{#2}
\else
\ifnum\value{hcode-counter} > 0
\m{#1~ #2}
\else
\HCode{[#1&\#126; #2]}
\fi
\fi
}

\newcommand{\mspdoc}[2]{
\ifpdf
%[\href{http://www.cycling74.com/docs/max5/refpages/msp-ref/#1~.html}{#1}$\sim$~#2]
[\href{http://cycling74.com/docs/max6/dynamic/c74_docs.html\##1~}{#1}$\sim$~#2]
\else
%[\href{http://www.cycling74.com/docs/max5/refpages/msp-ref/#1~.html}{#1}\HCode{&\#126;}~#2]
[\href{http://cycling74.com/docs/max6/dynamic/c74_docs.html\##1~}{#1}\HCode{&\#126;}~#2]
\fi
}

% message box
\newcommand{\msg}[1]{
\ifpdf
$\ullcorner\textrm{#1}\ulrcorner$
\else
\ifnum\value{hcode-counter} > 0
<table style="display:inline-table; vertical-align:-25\%" cellpadding="0" cellspacing="0" border="0">
<tr>
<td style="font-size:75\%;">\&\#x2308;</td>
<td rowspan="2">#1</td>
<td style="font-size:75\%;">\&\#x2309;</td>
</tr>
<tr>
<td style="font-size:75\%;">\&\#x230a;</td>
<td style="font-size:75\%;">\&\#x230b;</td>
</tr>
</table>
\else
\stepcounter{hcode-counter}
\HCode{
<table style="display:inline-table; vertical-align:-25\%" cellpadding="0" cellspacing="0" border="0">
<tr>
<td style="font-size:75\%;">&\#x2308;</td>
<td rowspan="2">#1</td>
<td style="font-size:75\%;">&\#x2309;</td>
</tr>
<tr>
<td style="font-size:75\%;">&\#x230a;</td>
<td style="font-size:75\%;">&\#x230b;</td>
</tr>
</table>
}
\setcounter{hcode-counter}{0}
\fi
\fi
}

% variable
\newcommand{\var}[1]{
\ifpdf
$<$#1$>$
\else
\ifnum\value{hcode-counter} > 0
&lt#1&gt
\else
\stepcounter{hcode-counter}
\HCode{&lt#1&gt}
\setcounter{hcode-counter}{0}
\fi
\fi
}

% insert a screen shot
\newcommand{\sshot}[3]{
\ifpdf
\begin{center}\includegraphics[width=#1]{#3}\end{center}
\else
\HCode{<div style="text-align: center;"><img src="\imagedir#3.png" width=#2></div>}
\fi
}

\newcommand{\gt}{Gametrak}


\graphicspath{{\imagedir}}

%\title{CNMAT Max/MSP Summer Class 2011\\Lab Assignment 4}
\ctitle{4}
\date{\today}

\begin{document}
\maketitle

\section*{Summary}
In \autoref{sec:envelopes} of this lesson, you will use the tool for managing 
envelopes that you worked on in lesson 3 to shape the amplitude envelope
of a simple \href{http://en.wikipedia.org/wiki/Frequency_modulation}{Frequency Modulation} (FM) 
synthesizer.  In \autoref{sec:buffer}, you will load soundfiles into
buffers and play them back, adjusting the speed of playback and loop points.

\subsection*{Topics}
Data storage and recall, envelopes, sound file buffers and playback.

\subsection*{Objects Introduced}
%objects: buffer~, info~, loadmess, loadbang, key, select, groove~, sig~, line~, 
\mspdoc{buffer}{}, \mspdoc{info}{}, \mdoc{loadmess}{}, \mdoc{loadbang}{}, \mdoc{key}{},
\mdoc{select}{}, \mspdoc{groove}{}, \mspdoc{sig}{}, \mspdoc{line}{}

\subsection*{Relevant Tutorials}
\subsubsection*{Basic}
\begin{enumerate}
\item \maxtut{basicchapter18}{Data Collections}
\end{enumerate}
\subsubsection*{MSP Topic}
\begin{enumerate}
%msp tutorial 4, 6, 7, 10 ,11, 14, 17, 23, 24
\item \msptut{mspchapter04}{Routing Signals}
\item \msptut{mspchapter06}{A Review of Fundamentals}
\item \msptut{mspchapter07}{Additive Synthesis}
\item \msptut{mspchapter10}{Vibrato and FM}
\item \msptut{mspchapter11}{Frequency Modulation}
\item \msptut{mspchapter14}{Playback with Loops}
\item \msptut{mspchapter17}{Sampling Review}
\item \msptut{mspchapter23}{Viewing Signal Data}
\item \msptut{mspchapter24}{Oscilloscope}
\end{enumerate}

\section{Envelopes}\label{sec:envelopes}

We will use the envelope tool you built yesterday to control the amplitude of the FM
synthesizer in today's exercise.

\begin{enumerate}
\item Copy the the envelope management system (``env\_store\_recall'', along with all of its supporting objects) that you made in the previous lesson and paste them into today's patch (lesson\_4\_fm.maxpat).  
\item Read in the text file containing the envelopes you created by sending
\m{env\_store\_recall}{} an \msg{import} message.
Make sure you verify that the file was loaded correctly by sending the \msg{edit} message
to \m{env\_store\_recall}{}.
\item Create a \mdoc{button}{} object, and connect its outlet to the inlet of the \mdoc{function}{} object.
\item Create a \mspdoc{sig}{} object, and connect its outlet to the first inlet of \msp{simplefm}{}.
\item Connect the 2nd outlet (from the left) of the \mdoc{function}{} object to the left inlet of a \mspdoc{line}{} object.
\item Connect the left outlet of the \mspdoc{line}{} object to the right inlet of the \msp{*}{1.0} (\mspdoc{times}{}) object just above the \mspdoc{gain}{} object at the bottom of the patch.
\item Now turn on the audio by clicking on the speaker icon at the bottom of the patch and 
\textbf{slowly} turn up the volume.
\item Click the \mdoc{button}{} connected to the inlet of the \mdoc{function}{} object.  The volume of the
FM synthesizer should now be controlled by the shape of the envelope.
\end{enumerate}

Your patch should now look roughly like this:
\sshot{5in}{800}{fm_done}

\subsection{Extra}

For those of you who make it through the first part of the assignment easily, try the following extra
tasks:

\begin{enumerate}
\item Use the \mdoc{key}{} object in conjunction with the \mdoc{select}{} object to control the envelope
with the space bar.
\item The FM synthesizer has three parameters controlled by signals in the three inlets.  These
parameters are, from left to right, the
\begin{inparaenum}[(1)]
  \item carrier frequency (the pitch),
  \item harmonicity ratio (controls how consonant or dissonant the tone is)
  \item modulation index (controls the brightness of the tone---similar to the effect of adding \mspdoc{tanh}{}
to our synthesizer in lesson 2).
\end{inparaenum}

Add envelope control to the three parameters discussed above.  (HINT:  you will need to scale the output
of the \mspdoc{line}{} object appropriately for each parameter.)

\item Add a message to scale the duration of the envelopes in the \mdoc{function}{} object.  Check the help
file of the \mdoc{function}{} object to see what message you can use to change the length of the function.

\item Rewire the patch so that one of the parameters of the \gt{} controls the position in the envelope
(i.e., /x the position of the joystick, for example, might represent the position along the x-axis
of the \mdoc{function}{} object.


\end{enumerate}













\section{Sound file playback with \mspdoc{buffer}{}}\label{sec:buffer}

\begin{enumerate}

\item Open the patch called lesson\_4\_sampling.maxpat.

\item Start by creating a \mspdoc{buffer}{} object and give it a name as its first argument, e.g.
\mspdoc{buffer}{sound1}.  This name is the name by which other objects will access the sound file
that the \mspdoc{buffer}{} contains.

\item Send the \mspdoc{buffer}{} the \msg{replace} message---an open file dialog will appear
that you can use to navigate to a sound file.  Choose a short sound file in either AIFF or WAV
format (there are short sound files in the CNMAT MMJ-Depot in the media/audio directory).

\item Double-click on the \mspdoc{buffer}{} object---a small window showing the waveform should open up.
If the window is empty, check the Max window for errors.

\item\label{groove-refer} Now, we need to tell the \mspdoc{groove}{} object to {\em refer} to your \mspdoc{buffer}{}.  
This can be done by either changing the first argument of \mspdoc{groove}{} to the same name as the 
buffer, or by sending \mspdoc{groove}{} the message \msg{set \var{buffername}}where \var{buffername}
is the first argument of the \mspdoc{buffer}{}.

\item Now, turn the audio system on by clicking on the speaker icon at the bottom of the patch
and \textbf{slowly} bring up the \mspdoc{gain}{} slider.

\item \mspdoc{groove}{} requires two things in order for it to play back a sound that it is referring to:
\begin{enumerate}
\item the speed of playback (as a signal), and
\item the point in the sound file from which to start (a message).
\end{enumerate}
A signal object \mspdoc{sig}{1.0} has already been connected to \mspdoc{groove}{} for you---this 
is the speed of playback.  Now, connect a message box to \mspdoc{groove}{} and send it the message
\msg{0}.  You should hear the entirety of your sound file played at ``normal'' speed.

\item Referring to the \mspdoc{groove}{} help file if you need to, see if you can figure out how to make the sound
play back at twice the speed, or at half the speed.  

\item Use the \gt{} to control the speed of the playback.

\end{enumerate}














\subsection{Starting from the middle of the file}

Now we would like to start playing from the middle of the file.  In order to do that, we need
to know the length of the sound file in milliseconds.

\begin{enumerate}
\item Start by creating an object called \mspdoc{info}{} that refers to your \mspdoc{buffer}{} 
(see step \ref{groove-refer} in the previous section).

\item Open the help file for \mspdoc{info}{} to see what the different outlets do.  Note the one 
that tells us the duration of the sound loaded into the \mspdoc{buffer}{} to which
the object refers.  Back in your patch, connect a number box to that outlet and send \mspdoc{info}{}
a bang.  You should now see the duration of your buffer in the number box.

\item We now have the duration of our sound file---see if you can 
finish wiring up the patch so that you can start the sample playback from anywhere you like.
\end{enumerate}











\subsubsection{Extra}

\begin{enumerate}
\item Use a \mdoc{multislider}{} to set the start point in the sample.  Have the \mdoc{multislider}{}
output values from 0--1 by changing the range in the inspector, and then multiply by the 
total duration of the sound file in milliseconds.

\item Use one of the parameters from the \gt{} to set the start position.  (HINT:  since
the \gt{} is constantly sending out values, you will have to store them and trigger them 
with something, e.g. the space bar, or the mouse.).
\end{enumerate}















\subsection{Looping}

\mspdoc{groove}{} has the ability to continually play a section of a sound file indefinitely.
The middle and right inlets control the loop start and stop points.  

\begin{enumerate}
\item Take two \mdoc{multislider}{}s and have them output values from 0--1 and scale them
by multiplying their output by the duration of the sound file in milliseconds.

\item Now send \mspdoc{groove}{} the message \msg{loop 1} followed by \msg{startloop}.  The
sound file should now be looping over the two loop points you have set.  
\end{enumerate}

\end{document}
